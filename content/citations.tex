
@ARTICLE{Henrick2008-zb,
  title    = "Remediation of the protein data bank archive",
  author   = "Henrick, Kim and Feng, Zukang and Bluhm, Wolfgang F and
              Dimitropoulos, Dimitris and Doreleijers, Jurgen F and Dutta,
              Shuchismita and Flippen-Anderson, Judith L and Ionides, John and
              Kamada, Chisa and Krissinel, Eugene and Lawson, Catherine L and
              Markley, John L and Nakamura, Haruki and Newman, Richard and
              Shimizu, Yukiko and Swaminathan, Jawahar and Velankar, Sameer and
              Ory, Jeramia and Ulrich, Eldon L and Vranken, Wim and Westbrook,
              John and Yamashita, Reiko and Yang, Huanwang and Young, Jasmine
              and Yousufuddin, Muhammed and Berman, Helen M",
  abstract = "The Worldwide Protein Data Bank (wwPDB; wwpdb.org) is the
              international collaboration that manages the deposition,
              processing and distribution of the PDB archive. The online PDB
              archive at ftp://ftp.wwpdb.org is the repository for the
              coordinates and related information for more than 47 000
              structures, including proteins, nucleic acids and large
              macromolecular complexes that have been determined using X-ray
              crystallography, NMR and electron microscopy techniques. The
              members of the wwPDB-RCSB PDB (USA), MSD-EBI (Europe), PDBj
              (Japan) and BMRB (USA)-have remediated this archive to address
              inconsistencies that have been introduced over the years. The
              scope and methods used in this project are presented.",
  journal  = "Nucleic Acids Res.",
  volume   =  36,
  number   = "Database issue",
  pages    = "D426--33",
  month    =  jan,
  year     =  2008,
  language = "en"
}

@ARTICLE{Yockey2019-pb,
  title    = "Cell Envelope Integrity and Capsule Characterization of
              Rhodotorula mucilaginosa Strains from Clinical and Environmental
              Sources",
  author   = "Yockey, Johnathan and Andres, Luke and Carson, Moleigh and Ory,
              Jeramia J and Reese, Amy J",
  abstract = "Rhodotorula yeasts are pink, encapsulated basidiomycetes isolated
              from a variety of environments and clinical settings. They are
              increasingly linked with disease, particularly central venous
              catheter infections and meningitis, in immunocompromised
              patients. Eight clinical and eight environmental strains
              molecularly typed as Rhodotorula mucilaginosa were compared to
              six Cryptococcus neoformans strains for phenotypic variability.
              Growth on cell integrity-challenging media suggested that R.
              mucilaginosa cells possess differences in signaling pathways,
              cell wall composition, or assembly and that their membranes are
              more susceptible to perturbations than those of C. neoformans All
              16 R. mucilaginosa strains produced urease, while none produced
              melanin with l-3,4-dihydroxyphenylalanine (l-DOPA) as a
              substrate. India ink staining reveals that clinical R.
              mucilaginosa capsules are larger than environmental capsules but
              that both are generally smaller than C. neoformans capsules. All
              R. mucilaginosa strains were resistant to fluconazole. Only two
              clinical strains were susceptible to voriconazole; all of the
              environmental strains were resistant. We generated an
              anticapsular antibody (Rh1) to R. mucilaginosa; Rh1 did not bind
              C. neoformans control strains, was specific to Rhodotorula
              species, and bound to all tested Rhodotorula strains. Binding
              assays performed with wheat germ agglutinin (WGA), concanavalin A
              (ConA), calcofluor white (CFW), and eosin Y dye (EY) cell surface
              probes suggested that chitin may be more accessible in R.
              mucilaginosa but that the total abundance of chitooligomers is
              less than in C. neoformans This report describes a novel reagent
              that can be used to identify Rhodotorula species and lays the
              foundation for future cell envelope composition
              analysis.IMPORTANCE Currently, there is very little known about
              the phenotypic variability within species of Rhodotorula strains
              and the role of their capsule. Cryptococcus neoformans has been
              considered the only encapsulated human fungal pathogen, but as
              more individuals come to live in states of immunocompromised
              health, they are more susceptible to fungal infections, including
              those by RhodotorulaR. mucilaginosa species are some of those
              most commonly associated with clinical infections. We wanted to
              know if clinical and environmental strains of R. mucilaginosa
              demonstrated disparate capsule phenotypes. With limited
              antifungal options available and clinical Rhodotorula spp. often
              resistant to common antifungal drugs such as fluconazole,
              caspofungin (1, 2), and voriconazole (2), a better understanding
              of the fungal biology could inform the design and use of future
              antifungal drugs. The generation of an antibody specific to
              Rhodotorula fungi could be a useful diagnostic tool, and this
              work presents the first mention of such in the literature.",
  journal  = "mSphere",
  volume   =  4,
  number   =  3,
  month    =  jun,
  year     =  2019,
  keywords = "Cryptococcus; Rhodotorula; capsule; cell wall integrity",
  language = "en"
}

@ARTICLE{McClelland2013-hj,
  title    = "The role of host gender in the pathogenesis of Cryptococcus
              neoformans infections",
  author   = "McClelland, Erin E and Hobbs, Letizia M and Rivera, Johanna and
              Casadevall, Arturo and Potts, Wayne K and Smith, Jennifer M and
              Ory, Jeramia J",
  abstract = "Cryptococcus neoformans (Cn) is a pathogenic yeast and the cause
              of cryptococcal meningitis. Prevalence of disease between males
              and females is skewed, with males having an increased incidence
              of disease. Based on the reported gender susceptibility
              differences to Cn in the literature, we used clinical isolates
              from Botswanan HIV-infected patients to test the hypothesis that
              different gender environments exerted different selective
              pressures on Cn. When we examined this data set, we found that
              men had significantly higher risk of death despite having
              significantly higher CD4(+) T lymphocyte counts upon admittance
              to the hospital. These observations suggested that Cn strains are
              uniquely adapted to different host gender environments and that
              the male immune response may be less efficient in controlling Cn
              infection. To discriminate between these possibilities, we tested
              whether there were phenotypic differences between strains
              isolated from males and females and whether there was an
              interaction between Cn and the host immune response. Virulence
              phenotypes showed that Cn isolates from females had longer
              doubling times and released more capsular glucoronoxylomannan
              (GXM). The presence of testosterone but not 17-$\beta$ estradiol
              was associated with higher levels of GXM release for a laboratory
              strain and 28 clinical isolates. We also measured phagocytic
              efficiency, survival of Cn, and amount of killing of human
              macrophages by Cn after incubation with four isolates. While
              macrophages from females phagocytosed more Cn than macrophages
              from males, male macrophages had a higher fungal burden and
              showed increased killing by Cn. These data are consistent with
              the hypothesis that differential interaction between Cn and
              macrophages within different gender environments contribute to
              the increased prevalence of cryptococcosis in males. This could
              be related to differential expression of cryptococcal virulence
              genes and capsule metabolism, changes in Cn phagocytosis and
              increased death of Cn-infected macrophages.",
  journal  = "PLoS One",
  volume   =  8,
  number   =  5,
  pages    = "e63632",
  month    =  may,
  year     =  2013,
  language = "en"
}

@ARTICLE{Yockey2019-cc,
  title     = "Cell Envelope Integrity and Capsule Characterization of
               Rhodotorula mucilaginosa Strains from Clinical and Environmental
               Sources",
  author    = "Yockey, Johnathan and Andres, Luke and Carson, Moleigh and Ory,
               Jeramia J and Reese, Amy J",
  abstract  = "Rhodotorula yeasts are pink, encapsulated basidiomycetes
               isolated from a variety of environments and clinical settings.
               They are increasingly linked with disease, particularly central
               venous catheter infections and meningitis, in immunocompromised
               patients. Eight clinical and eight environmental strains
               molecularly typed as Rhodotorula mucilaginosa were compared to
               six Cryptococcus neoformans strains for phenotypic variability.
               Growth on cell integrity-challenging media suggested that R.
               mucilaginosa cells possess differences in signaling pathways,
               cell wall composition, or assembly and that their membranes are
               more susceptible to perturbations than those of C. neoformans
               All 16 R. mucilaginosa strains produced urease, while none
               produced melanin with l-3,4-dihydroxyphenylalanine (l-DOPA) as a
               substrate. India ink staining reveals that clinical R.
               mucilaginosa capsules are larger than environmental capsules but
               that both are generally smaller than C. neoformans capsules. All
               R. mucilaginosa strains were resistant to fluconazole. Only two
               clinical strains were susceptible to voriconazole; all of the
               environmental strains were resistant. We generated an
               anticapsular antibody (Rh1) to R. mucilaginosa; Rh1 did not bind
               C. neoformans control strains, was specific to Rhodotorula
               species, and bound to all tested Rhodotorula strains. Binding
               assays performed with wheat germ agglutinin (WGA), concanavalin
               A (ConA), calcofluor white (CFW), and eosin Y dye (EY) cell
               surface probes suggested that chitin may be more accessible in
               R. mucilaginosa but that the total abundance of chitooligomers
               is less than in C. neoformans This report describes a novel
               reagent that can be used to identify Rhodotorula species and
               lays the foundation for future cell envelope composition
               analysis.IMPORTANCE Currently, there is very little known about
               the phenotypic variability within species of Rhodotorula strains
               and the role of their capsule. Cryptococcus neoformans has been
               considered the only encapsulated human fungal pathogen, but as
               more individuals come to live in states of immunocompromised
               health, they are more susceptible to fungal infections,
               including those by RhodotorulaR. mucilaginosa species are some
               of those most commonly associated with clinical infections. We
               wanted to know if clinical and environmental strains of R.
               mucilaginosa demonstrated disparate capsule phenotypes. With
               limited antifungal options available and clinical Rhodotorula
               spp. often resistant to common antifungal drugs such as
               fluconazole, caspofungin (1, 2), and voriconazole (2), a better
               understanding of the fungal biology could inform the design and
               use of future antifungal drugs. The generation of an antibody
               specific to Rhodotorula fungi could be a useful diagnostic tool,
               and this work presents the first mention of such in the
               literature.",
  journal   = "mSphere",
  publisher = "Am Soc Microbiol",
  volume    =  4,
  number    =  3,
  month     =  jun,
  year      =  2019,
  keywords  = "Cryptococcus; Rhodotorula; capsule; cell wall integrity",
  language  = "en"
}

@ARTICLE{Bose2003-fd,
  title    = "A yeast under cover: the capsule of Cryptococcus neoformans",
  author   = "Bose, Indrani and Reese, Amy J and Ory, Jeramia J and Janbon,
              Guilhem and Doering, Tamara L and Bose, I and Reese, A and Ory, J
              J and Janbon, G and Doering, T L",
  journal  = "Eukaryot. Cell",
  volume   =  2,
  number   =  4,
  pages    = "655--663",
  month    =  aug,
  year     =  2003,
  keywords = "Cell Wall; Cryptococcus neoformans; Extracellular Matrix;
              Polysaccharides; Signal Transduction"
}

@ARTICLE{Ory2004-kp,
  title    = "An efficiently regulated promoter system for Cryptococcus
              neoformans utilizing the {CTR4} promoter",
  author   = "Ory, Jeramia J and Griffith, Cara L and Doering, Tamara L",
  abstract = "Cryptococcus neoformans is an opportunistic fungal pathogen
              responsible for serious meningitis. Although many useful
              molecular tools have been developed for its study, there are
              currently few inducible promoters available for general use. To
              address this need, we have constructed expression plasmids
              incorporating upstream elements of the C. neoformans copper
              transporter gene CTR4, and tested them in C. neoformans serotypes
              A and D. In response to copper deprivation, these plasmids
              mediate high-level expression of a reporter protein. This
              expression can be completely repressed using physiologically low
              concentrations of copper. Notably, this new family of
              copper-sensing promoters demonstrates excellent expression in
              serotype A, contrasting with other available promoters. These
              plasmids therefore offer efficient and regulated expression for
              both serotypes A and D, and should be valuable tools for the C.
              neoformans research community.",
  journal  = "Yeast",
  volume   =  21,
  number   =  11,
  pages    = "919--926",
  month    =  aug,
  year     =  2004,
  keywords = "Cation Transport Proteins; Copper; Cryptococcus neoformans;
              Fungal Proteins; Gene Expression Regulation, Fungal; Plasmids;
              Promoter Regions, Genetic; Serotyping; Transcription Factors;
              Transformation, Genetic"
}

@ARTICLE{Bar-Peled2004-tg,
  title    = "Biosynthesis of {UDP-GlcA}, a key metabolite for capsular
              polysaccharide synthesis in the pathogenic fungus Cryptococcus
              neoformans",
  author   = "Bar-Peled, Maor and Griffith, Cara L and Ory, Jeramia J and
              Doering, Tamara L",
  abstract = "UDP-glucose dehydrogenase catalyses the conversion of UDP-glucose
              into UDP-GlcA, a critical precursor for glycan synthesis across
              evolution. We have cloned the gene encoding this important enzyme
              from the opportunistic pathogen Cryptococcus neoformans. In this
              fungus, UDP-GlcA is required for the synthesis of capsule
              polysaccharides, which in turn are essential for virulence. The
              gene was expressed in Escherichia coli and the 51.3-kDa
              recombinant protein from wild-type and five mutants was purified
              for analysis. The cryptococcal enzyme is strongly inhibited by
              UDP-xylose and NADH, has highest activity at pH 7.5 and
              demonstrates Km (app) values of 0.1 and 1.5 mM for NAD+ and
              UDP-glucose respectively. Its activity was significantly
              decreased by mutations in the putative sites of NAD+ and
              UDP-glucose binding. Unlike previously reported eukaryotic
              UDP-glucose dehydrogenases, which are hexamers, the cryptococcal
              enzyme is a dimer.",
  journal  = "Biochem. J",
  volume   =  381,
  number   = "Pt 1",
  pages    = "131--136",
  month    =  jul,
  year     =  2004,
  language = "en"
}

@ARTICLE{Ory1998-bg,
  title    = "Structural characterization of two synthetic catalysts based on
              adipocyte lipid-binding protein",
  author   = "Ory, J J and Mazhary, A and Kuang, H and Davies, R R and
              Distefano, M D and Banaszak, L J",
  abstract = "Adipocyte lipid-binding protein (ALBP) is a small (14.5 kDa)
              10-stranded beta-barrel protein found in mammalian fat cells. The
              crystal structures of various holo-forms of ALBP have been solved
              and show the fatty acid ligand bound in a large (approximately
              400 A3) cavity isolated from bulk solvent. Examination of the
              cavity suggests that it would be a good site for the creation of
              an artificial catalyst, as numerous well defined crystal
              structures of ALBP are available and past studies have shown the
              conformation to be reasonably tolerant to modification and
              mutagenesis. Previous work has shown ALBP to be a good protein
              scaffold for exploring enantio- and stereoselective reactions;
              two constructs, ALBP attached to either a pyridoxamine or a
              phenanthroline group at C117, have been chemically characterized.
              Both modified proteins have been crystallized and their
              structures solved and refined. The X-ray models have been used to
              examine the origin of the chiral selectivity seen in the
              products. It is apparent that these covalent adducts reduce the
              internal cavity volume, sterically limiting substrate
              interactions with the reactive groups, as well as solvent access
              to potential intermediates in the reaction pathway.",
  journal  = "Protein Eng.",
  volume   =  11,
  number   =  4,
  pages    = "253--261",
  month    =  apr,
  year     =  1998,
  language = "en"
}

@ARTICLE{Thompson1999-ri,
  title    = "The liver fatty acid binding protein--comparison of cavity
              properties of intracellular lipid-binding proteins",
  author   = "Thompson, J and Ory, J and Reese-Wagoner, A and Banaszak, L",
  abstract = "The crystal and solution structures of all of the intracellular
              lipid binding proteins (iLBPs) reveal a common beta-barrel
              framework with only small local perturbations. All existing
              evidence points to the binding cavity and a poorly delimited
              'portal' region as defining the function of each family member.
              The importance of local structure within the cavity appears to be
              its influence on binding affinity and specificity for the lipid.
              The portal region appears to be involved in the regulation of
              ligand exchange. Within the iLBP family, liver fatty acid binding
              protein or LFABP, has the unique property of binding two fatty
              acids within its internalized binding cavity rather than the
              commonly observed stoichiometry of one. Furthermore, LFABP will
              bind hydrophobic molecules larger than the ligands which will
              associate with other iLBPs. The crystal structure of LFABP
              contains two bound oleate molecules and provides the explanation
              for its unusual stoichiometry. One of the bound fatty acids is
              completely internalized and has its carboxylate interacting with
              an arginine and two serines. The second oleate represents an
              entirely new binding mode with the carboxylate on the surface of
              LFABP. The two oleates also interact with each other. Because of
              this interaction and its inner location, it appears the first
              oleate must be present before the second more external molecule
              is bound.",
  journal  = "Mol. Cell. Biochem.",
  volume   =  192,
  number   = "1-2",
  pages    = "9--16",
  month    =  feb,
  year     =  1999
}

@ARTICLE{Ory2008-mx,
  title   = "Moving Pictures: Using Chimera to make molecular multimedia for
             the classroom",
  author  = "Ory, Jeramia and Ory, Jeramia",
  journal = "PDB Newsletter",
  month   =  jan,
  year    =  2008
}

@ARTICLE{Ory1992-qg,
  title    = "Physiological consequences of {DnaK} and {DnaJ} overproduction in
              Escherichia coli",
  author   = "Ory, J and Blum, P and Ory, J and Bauernfeind, J and Krska, J and
              Blum, P and Bauernfeind, J and Krska, J",
  abstract = "The physiological consequences of molecular chaperone
              overproduction in Escherichia coli are presented. Constitutive
              overproduction of DnaK from a multicopy plasmid containing large
              chromosomal fragments spanning the dnaK region resulted in
              plasmid instability. Co-overproduction of DnaJ with DnaK
              stabilized plasmid levels. To examine the effects of altered
              levels of DnaK and DnaJ in a more specific manner, an inducible
              expression system for dnaK and dnaJ was constructed and
              characterized. Differential rates of DnaK synthesis were
              determined by quantitative Western blot (immunoblot) analysis.
              Moderate levels of DnaK overproduction resulted in a defect in
              cell septation and formation of cell filaments, but
              co-overproduction of DnaJ overcame this effect. Further increases
              in the level of DnaK terminated culture growth despite increased
              levels of DnaJ. DnaK overproduction was found to be
              bacteriocidal, and this effect was also partially suppressed by
              DnaJ. The bacteriocidal effect was apparent only with cultures
              which were allowed to enter stationary phase, indicating that
              DnaK toxicity is growth phase dependent.",
  journal  = "J. Bacteriol.",
  volume   =  174,
  number   =  22,
  pages    = "7436--7444",
  month    =  nov,
  year     =  1992
}

@ARTICLE{Ory2004-yy,
  title     = "Scientists and societies. Two-body problem",
  author    = "Ory, Jeramia J",
  journal   = "Nature",
  publisher = "Nature Publishing Group",
  volume    =  429,
  number    =  6993,
  pages     = "788",
  month     =  jun,
  year      =  2004
}

@ARTICLE{McClelland2013-zd,
  title     = "The Role of Host Gender in the Pathogenesis of Cryptococcus
               neoformans Infections",
  author    = "McClelland, Erin E and Hobbs, Letizia M and Rivera, Johanna and
               Casadevall, Arturo and Potts, Wayne K and Smith, Jennifer M and
               Ory, Jeramia J",
  abstract  = "Abstract Cryptococcus neoformans (Cn) is a pathogenic yeast and
               the cause of cryptococcal meningitis. Prevalence of disease
               between males and females is skewed, with males having an
               increased incidence of disease. Based on the reported gender
               susceptibility ...",
  journal   = "PLoS One",
  publisher = "Public Library of Science",
  volume    =  8,
  number    =  5,
  pages     = "e63632",
  month     =  may,
  year      =  2013
}

@ARTICLE{Burkhardt2006-rp,
  title     = "A biocurator perspective: annotation at the Research
               Collaboratory for Structural Bioinformatics Protein Data Bank",
  author    = "Burkhardt, Kyle and Ory, Jeramia J and Schneider, Bohdan and
               Ory, Jeramia J and Burkhardt, Kyle and Schneider, Bohdan",
  journal   = "PLoS Comput. Biol.",
  publisher = "Public Library of Science",
  volume    =  2,
  number    =  10,
  pages     = "e99",
  month     =  oct,
  year      =  2006
}

@ARTICLE{Ory1997-wq,
  title    = "Biochemical and crystallographic analyses of a portal mutant of
              the adipocyte lipid-binding protein",
  author   = "Ory, J and Kane, C and Simpson, M and Banaszak, L and Bernlohr, D
              and Ory, J and Kane, C and Simpson, M and Banaszak, L and
              Bernlohr, D and Kane, C D and Simpson, M A and Banaszak, L J and
              Bernlohr, D A",
  abstract = "A number of crystallographic studies of the adipocyte
              lipid-binding protein have established that the fatty
              acid-binding site is within an internalized water-filled cavity.
              The same studies have also suggested the existence of a region
              physically distinct from the fatty acid-binding site which
              connects the cavity of the protein with the external solvent,
              hereafter referred to as the portal. In an effort to examine the
              portal region, we have used site-directed mutagenesis to
              introduce the mutations V32D/F57H into the murine ALBP cDNA.
              Mutant protein has been isolated, crystallized, and its stability
              and binding properties studied by biochemical methods. As
              assessed by guanidine-HCl denaturation, the mutant form exhibited
              a slight overall destabilization relative to the wild-type
              protein under both acid and alkaline conditions. Accessibility to
              the cavity in both the mutant and wild-type proteins was observed
              by stopped-flow analysis of the modification of a cavity residue,
              Cys117, by the sulfhydryl reactive agent 5,
              5'-dithiobis(2-nitrobenzoic acid) at pH 8.5. Cys117 of V32D/F57H
              ALBP was modified 7-fold faster than the wild-type protein. The
              ligand binding properties of both the V32D/F57H mutant and
              wild-type proteins were analyzed using a fluorescent probe at pH
              6.0 and 8.0. The apparent dissociation constants for
              1-anilinonaphthalene-8-sulfonic acid were approximately 9-10-fold
              greater than the wild-type protein, independent of pH. In
              addition, there is a 6-fold increase in the Kd for oleic acid for
              the portal mutant relative to the wild-type at pH 8.0. To study
              the effect of pH on the double mutant, it was crystallized and
              analyzed in two distinct space groups at pH 4.5 and 6.4. While in
              general the differences in the overall main chain conformations
              are negligible, changes were observed in the crystallographic
              structures near the site of the mutations. At both pH values, the
              mutant side chains are positioned somewhat differently than in
              wild-type protein. To ensure that the mutations had not altered
              ionic conditions near the binding site, the crystallographic
              coordinates were used to monitor the electrostatic potentials
              from the head group site to the positions near the portal region.
              The differences in the electrostatic potentials were small in all
              regions, and did not explain the differences in ligand affinity.
              We present these results within the context of fatty acid binding
              and suggest lipid association is more complex than that described
              within a single equilibrium event.",
  journal  = "J. Biol. Chem.",
  volume   =  272,
  number   =  15,
  pages    = "9793--9801",
  month    =  apr,
  year     =  1997,
  keywords = "Anilino Naphthalenesulfonates; Animals; Binding Sites; Carrier
              Proteins; Crystallography- X-Ray; Fatty Acid-Binding Proteins;
              Fatty Acids; Fluorescent Dyes; Hydrogen-Ion Concentration; Mice;
              Models- Molecular; Molecular Sequence Data; Mutagenesis-
              Site-Directed; Myelin P2 Protein; Neoplasm Proteins; Nerve Tissue
              Proteins; Oleic Acid; Protein Conformation; Protein Structure-
              Secondary; Static Electricity"
}

@ARTICLE{Henrick2008-lc,
  title    = "Remediation of the protein data bank archive",
  author   = "Henrick, Kim and Feng, Zukang and Bluhm, Wolfgang F and
              Dimitropoulos, Dimitris and Doreleijers, Jurgen F and Dutta,
              Shuchismita and Flippen-Anderson, Judith L and Ionides, John and
              Kamada, Chisa and Krissinel, Eugene and Lawson, Catherine L and
              Markley, John L and Nakamura, Haruki and Newman, Richard and
              Shimizu, Yukiko and Swaminathan, Jawahar and Velankar, Sameer and
              Ory, Jeramia J and Ulrich, Eldon L and Vranken, Wim and
              Westbrook, John and Yamashita, Reiko and Yang, Huanwang and
              Young, Jasmine and Yousufuddin, Muhammed and Berman, Helen M",
  abstract = "The Worldwide Protein Data Bank (wwPDB; wwpdb.org) is the
              international collaboration that manages the deposition,
              processing and distribution of the PDB archive. The online PDB
              archive at ftp://ftp.wwpdb.org is the repository for the
              coordinates and related information for more than 47 000
              structures, including proteins, nucleic acids and large
              macromolecular complexes that have been determined using X-ray
              crystallography, NMR and electron microscopy techniques. The
              members of the wwPDB-RCSB PDB (USA), MSD-EBI (Europe), PDBj
              (Japan) and BMRB (USA)-have remediated this archive to address
              inconsistencies that have been introduced over the years. The
              scope and methods used in this project are presented.",
  journal  = "Nucleic Acids Res.",
  volume   =  36,
  number   = "Database issue",
  pages    = "D426--33",
  month    =  jan,
  year     =  2008
}

@ARTICLE{Terradot2004-dv,
  title    = "Biochemical characterization of protein complexes from the
              Helicobacter pylori protein interaction map: strategies for
              complex formation and evidence for novel interactions within type
              {IV} secretion systems",
  author   = "Terradot, Laurent and Durnell, Nathan and Li, Min and Li, Ming
              and Ory, Jeremiah and Labigne, Agnes and Legrain, Pierre and
              Colland, Frederic and Waksman, Gabriel and Terradot, Laurent and
              Durnell, Nathan and Li, Min and Li, Ming and Ory, Jeremiah and
              Labigne, Agnes and Legrain, Pierre and Colland, Frederic and
              Waksman, Gabriel",
  abstract = "We have investigated a large set of interactions from the
              Helicobacter pylori protein interaction map previously identified
              by high-throughput yeast two-hybrid (htY2H)-based methods. This
              study had two aims: i) to validate htY2H as a source of
              protein-protein interaction complexes for high-throughput
              biochemical and structural studies of the H. pylori interactome;
              and ii) to validate biochemically interactions shown by htY2H to
              involve components of the H. pylori type IV secretion systems.
              Thus, 17 interactions involving 31 proteins and protein fragments
              were studied, and a general strategy was designed to produce
              protein-interacting partners for biochemical and structural
              characterization. We show that htY2H is a valid source of
              protein-protein complexes for high-throughput proteome-scale
              characterization of the H. pylori interactome, because 76\% of
              the interactions tested were confirmed biochemically. Of the
              interactions involving type IV secretion proteins, three could be
              confirmed. One interaction is between two components of the type
              IV secretion apparatus, ComB10 and ComB4, which are VirB10 and
              VirB4 homologs, respectively. Another interaction is between a
              type IV component (HP0525, a VirB11 homolog) and a non-type IV
              secretion protein (HP01451), indicating that proteins other than
              the core VirB (1-11)-VirD4 proteins may play a role in type IV
              secretion. Finally, a third interaction was biochemically
              confirmed between CagA, a virulence factor secreted by the type
              IV secretion system encoded by the Cag pathogenicity island, and
              a non-type IV secretion protein, HP0496.",
  journal  = "Mol. Cell. Proteomics",
  volume   =  3,
  number   =  8,
  pages    = "809--819",
  month    =  aug,
  year     =  2004
}

@ARTICLE{Ory1999-pz,
  title    = "Studies of the ligand binding reaction of adipocyte lipid binding
              protein using the fluorescent probe 1,
              8-anilinonaphthalene-8-sulfonate",
  author   = "Ory, J J and Banaszak, L J",
  abstract = "The fluorescent probe anilinonaphthalene-8-sulfonate binds to
              adipocyte lipid binding protein at a site that competes with
              normal physiological ligands, such as fatty acids. Binding to the
              protein is accompanied by a relatively large increase in
              fluorescent intensity. To correlate the major change in optical
              properties and to determine the mechanism of competitive
              inhibition with fatty acids, the crystal structure of the protein
              with the bound fluorophore has been determined. In addition, the
              thermodynamic contributions to the binding reaction have been
              studied by titration calorimetry. Because the binding site is in
              a relatively internal position, kinetic studies have also been
              carried out to determine k(on). The results indicate that binding
              is not accompanied by any major conformational change. However,
              the negatively charged sulfonate moiety is not positioned the
              same as the carboxylate of fatty acid ligands as determined in
              previous studies. Nonetheless, the binding reaction is still
              driven by enthalpic effects. As judged by the crystallographic
              structure, a significant amount of the surface of the fluorophore
              is no longer exposed to water in the bound state.",
  journal  = "Biophys. J.",
  volume   =  77,
  number   =  2,
  pages    = "1107--1116",
  month    =  aug,
  year     =  1999,
  keywords = "Anilino Naphthalenesulfonates; Animals; Biophysical Phenomena;
              Biophysics; Carrier Proteins; Fatty Acid-Binding Proteins; Fatty
              Acids; Fluorescent Dyes; Kinetics; Ligands; Mice; Myelin P2
              Protein; Neoplasm Proteins; Nerve Tissue Proteins; Protein
              Conformation; Static Electricity; Thermodynamics"
}
